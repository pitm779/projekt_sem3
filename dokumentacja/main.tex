\documentclass{article}
\usepackage{amsmath}
\usepackage{listings}
\usepackage{graphicx}
\usepackage{float}
\usepackage{booktabs}
\usepackage{geometry}
\usepackage{hyperref}
\usepackage[section]{placeins}
\usepackage[T1]{fontenc}

% Ustawienia marginesów
\geometry{top=3cm,bottom=2cm,left=2cm,right=2cm}

\title{Dokumentacja Projektu Zaliczeniowego\\ \huge Baza danych dla firmy Wombat Grylls sp. z o.o.}
\author{Kacper Daniel, Tadeusz Jagniewski, Paweł Karwecki, Piotr Marciniak, Konrad Mądry}
\date{\today}

\begin{document}

\maketitle

\newpage

\tableofcontents
\newpage

\section{Spis użytych technologii}
\begin{itemize}
    \item System zarządzania bazą danych: MariaDB
    \item Generowanie danych: Python - pymysql, faker, geopy, numpy, pandas
    \item Analiza danych i raport: R - 
    \item Format raportu: PDF
\end{itemize}

\section{Lista plików i opis ich zawartości}
\begin{itemize}
    \item \texttt{schemat.erd} - Diagram ERD bazy danych.
    \item \texttt{wszystko.py} - Plik uruchamiający skrypty odpowiadające za 1) tworzenie pustych tabel - tabele.py (z pliku tabele.sql); 2) wypełnianie tabeli kategorii - category\_fill.py; 3) wypełnianie pozostałych tabel - \\database\_filler\_template.py
    \item \texttt{requirements.txt} - Zawiera listę wszystkich pythonowych bibliotek wraz z wersjami niezbędnych do uruchomienia wszystkich skryptów
    \item \texttt{raport\_team03.pdf} - Raport
    \item \texttt{raport\_team03.Rmd} - Plik R generujący raport
    \item \texttt{lista\_nazwisk.csv i lista\_imion.csv} - Pliki służące do generowania imion i nazwisk 
    \item \texttt{Adresy\_StanNa20250119.csv} - Plik związany z analizą
    \item \texttt{generowanie\_tabel.py} - Plik ze wszystkimi funkcjami generującymi dane
\end{itemize}

\section{Kolejność i sposób uruchamiania plików}
\begin{enumerate}
    \item \texttt{}
    \item \texttt{}
\end{enumerate}

\section{Schemat projektu bazy danych}

\newpage

\section{Lista zależności funkcyjnych dla każdej relacji}

\subsection{Tabela: Address}
\begin{itemize}
    \item address\_id $\rightarrow$ address, postal\_code, city
\end{itemize}
Kilka miejscowości może mieć ten sam kod pocztowy, więc nie ma tutaj zależności tranzytywnych.
\subsection{Tabela: Costs}
\begin{itemize}
    \item cost\_id $\rightarrow$ trip\_id, name, amount
    \item trip\_id $\rightarrow$ trip\_id (odwołanie do tabeli trips)
\end{itemize}
\subsection{Tabela: Customers}
\begin{itemize}
    \item customer\_id $\rightarrow$ address\_id, first\_name, last\_name, email, phone\_number, amount, birth\_date, ICE\_number
    \item address\_id $\rightarrow$ address\_id (odwołanie do tabeli address)
\end{itemize}
Zakładamy, że numer telefonu i email nie muszą być unikalne.
\subsection{Tabela: Payment}
\begin{itemize}
    \item payment\_id $\rightarrow$ customer\_id, staff\_id, trip\_id, payment\_date, amount
    \item customer\_id $\rightarrow$ customer\_id (odwołanie do tabeli customers)
    \item staff\_id $\rightarrow$ staff\_id (odwołanie do tabeli staff)
    \item trip\_id $\rightarrow$ trip\_id (odwołanie do tabeli trips)
\end{itemize}
\subsection{Tabela: Staff}
\begin{itemize}
    \item staff\_id $\rightarrow$ address\_id, first\_name, last\_name, salary, email, hire\_date, birth\_date
    \item address\_id $\rightarrow$ address\_id (odwołanie do tabeli address)
\end{itemize}
Ponownie, email nie musi być unikalny (dwóch pracowników może korzystać z tego samego adresu e-mail)
\subsection{Tabela: Trips}
\begin{itemize}
    \item trip\_id $\rightarrow$ category\_id, trip\_name, cost\_to\_client, begin\_date, end\_date, abroad, creation\_date, description
    \item category\_id $\rightarrow$ category\_id (odwołanie do tabeli trip\_category)
\end{itemize}
\subsection{Tabela: Trip\_category}
\begin{itemize}
    \item category\_id $\rightarrow$ category\_name
\end{itemize}

\newpage

\section{Uzasadnienie, że baza jest w EKNF}
Baza danych została zaprojektowana zgodnie z zasadami EKNF. Każda tabela posiada jednoznaczny klucz główny. Brak zależności przechodnich. \\Kilku klientów może podać ten sam adres stąd przeniesienie tych informacji do osobnej tabeli.

\section{Opis trudności podczas realizacji projektu}

\end{document}
