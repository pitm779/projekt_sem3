\documentclass{article}
\usepackage{amsmath}
\usepackage{listings}
\usepackage{graphicx}
\usepackage{float}
\usepackage{booktabs}
\usepackage{geometry}
\usepackage{hyperref}
\usepackage[section]{placeins}
\usepackage[T1]{fontenc}

% Ustawienia marginesów
\geometry{top=3cm,bottom=2cm,left=2cm,right=2cm}

\title{Dokumentacja Projektu Zaliczeniowego\\ \huge Baza danych dla firmy Wombat Grylls sp. z o.o.}
\author{Autorzy}
\date{\today}

\begin{document}

\maketitle

\newpage

\tableofcontents
\newpage

\section{Spis użytych technologii}
\begin{itemize}
    \item System zarządzania bazą danych: MariaDB
    \item Język programowania: Python (do generowania danych)
    \item Narzędzie do analizy i raportowania: 
    \item Narzędzia do generowania wykresów: 
    \item Format raportu: PDF/HTML
\end{itemize}

\section{Lista plików i opis ich zawartości}
\begin{itemize}
    \item \texttt{schemat.erd} - Diagram ERD bazy danych.
\end{itemize}

\section{Kolejność i sposób uruchamiania plików}
\begin{enumerate}
    \item 
\end{enumerate}

\section{Schemat projektu bazy danych}
\begin{figure}[h!]
\centering
%\includegraphics[width=0.8\textwidth]{schema.png}  % Podmień na odpowiednią ścieżkę do pliku z schematem
\caption{Schemat bazy danych firmy Wombat Grylls sp. z o.o.}
\end{figure}

\newpage

\section{Lista zależności funkcyjnych dla każdej relacji}

\subsection{Tabela: Address}
\begin{itemize}
    \item address\_id $\rightarrow$ address, postal\_code, city
\end{itemize}
\subsection{Tabela: Costs}
\begin{itemize}
    \item cost\_id $\rightarrow$ trip\_id, name, amount
    \item trip\_id $\rightarrow$ trip\_id (odwołanie do tabeli trips)
\end{itemize}
\subsection{Tabela: Customers}
\begin{itemize}
    \item customer\_id $\rightarrow$ address\_id, first\_name, last\_name, email, phone\_number, amount, birth\_date, ICE\_number
    \item address\_id $\rightarrow$ address\_id (odwołanie do tabeli address)
\end{itemize}
\subsection{Tabela: Payment}
\begin{itemize}
    \item payment\_id $\rightarrow$ customer\_id, staff\_id, trip\_id, payment\_date, amount
    \item customer\_id $\rightarrow$ customer\_id (odwołanie do tabeli customers)
    \item staff\_id $\rightarrow$ staff\_id (odwołanie do tabeli staff)
    \item trip\_id $\rightarrow$ trip\_id (odwołanie do tabeli trips)
\end{itemize}
\subsection{Tabela: Staff}
\begin{itemize}
    \item staff\_id $\rightarrow$ address\_id, first\_name, last\_name, salary, email, hire\_date, birth\_date
    \item address\_id $\rightarrow$ address\_id (odwołanie do tabeli address)
\end{itemize}
\subsection{Tabela: Trips}
\begin{itemize}
    \item trip\_id $\rightarrow$ category\_id, trip\_name, cost\_to\_client, begin\_date, end\_date, abroad, creation\_date, description
    \item category\_id $\rightarrow$ category\_id (odwołanie do tabeli trip\_category)
\end{itemize}
\subsection{Tabela: Trip\_category}
\begin{itemize}
    \item category\_id $\rightarrow$ category\_name
\end{itemize}

\newpage

\section{Uzasadnienie, że baza jest w EKNF}
Baza danych została zaprojektowana zgodnie z zasadami EKNF (Efektywnej Normalnej Formy Encji). W każdej tabeli spełnione są następujące warunki:
\begin{itemize}
    \item Każda tabela posiada jednoznaczny klucz główny.
    \item Brak zależności przechodnich - wszystkie zależności funkcyjne są określone przez klucze obce.
    \item Zależności funkcyjne są zgodne z zasadami normalizacji, eliminując redundancję.
\end{itemize}

\section{Opis trudności podczas realizacji projektu}

\end{document}
